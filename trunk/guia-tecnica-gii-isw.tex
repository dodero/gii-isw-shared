% Created 2012-03-27 mar 13:13
\documentclass[11pt]{article}
\usepackage[utf8]{inputenc}
\usepackage[T1]{fontenc}
\usepackage{fixltx2e}
\usepackage{graphicx}
\usepackage{longtable}
\usepackage{float}
\usepackage{wrapfig}
\usepackage{soul}
\usepackage{textcomp}
\usepackage{marvosym}
\usepackage{wasysym}
\usepackage{latexsym}
\usepackage{amssymb}
\usepackage{hyperref}
\tolerance=1000
\usepackage{color}
\usepackage{listings}
\usepackage{geometry}\geometry{hmargin=2cm,vmargin=3cm}
\usepackage{fancyhdr}\pagestyle{fancy}\lhead{Guía Técnica de ISW}\rhead{Curso 2011/2012}\cfoot{Página \thepage}
\usepackage{hyperref}\hypersetup{pdfborder={0 0 0},colorlinks,linkcolor=blue,urlcolor=blue}
\lstset{language=java,frame=single, basicstyle=\footnotesize,breaklines=true,emphstyle=\textbf}
\providecommand{\alert}[1]{\textbf{#1}}


\title{Especialidad de Ingeniería del Software\\
Grado en Ingeniería Informática\\
Universidad de Cádiz\\
\textbf{Guía Técnica para las Prácticas}}
\author{Daniel Molina Cabrera <daniel.molina@uca.es> \and Juan Manuel Dodero <juanma.dodero@uca.es>}
\date{Curso 2011/2012}


\begin{document}


\maketitle


\setcounter{tocdepth}{2}
\tableofcontents
\vspace*{1cm}
\newpage
\setlength{\parindent}{0pt}
\setlength{\parskip}{1ex plus 0.5ex minus 0.2ex}


\section{Descripción del documento}
\label{sec-1}




Este documento se plantea como una guía técnica para la realización de las prácticas y el proyecto de las asignaturas de la especialidad de Ingeniería del Software en el Grado en Ingeniería Informática y en el segundo ciclo de Ingeniería Informática. Esta guía es válida para las siguientes asignaturas:

\begin{itemize}
   \item Diseño de Sistemas Software (DSS)
   \item Ingeniería Web (IW)
   \item Implementación e Implantación de Sistemas Software (IISS)
   \item \ldots
\end{itemize}

La idea de este documento es el de ofrecer una guía que permita a los alumnos/as de dichas asignaturas reducir los problemas técnicos y centrarse en la resolución de las prácticas.


\section{Lenguaje Java}
\label{sec-2}

Las prácticas se realizan en general empleando el lenguaje de Programación Java. Se ha elegido este lenguaje ya que es un lenguaje muy extendido, uno de los más usados en el desarrollo de aplicaciones web de servidor, así como en otros tipos de sistemas ---por ejemplo, las aplicaciones móbiles para Android se pueden programar usando Java.

Otros puntos a favor es su cercanía al lenguaje (C++) ya aprendido en asignaturas anteriores, y la existencia de excelentes Integrated Development Environments (IDE) libres que simplifican mucho el desarrollo.

Tal y como se describe en el apartado de \hyperref[sec-3]{IDEs}, se propone usar un IDE, para facilitar el trabajo. Además, cada proyecto debe crearse con una estructura estándar, independiente del IDE. Para ello se empleará la herramienta \emph{maven}, descrita en el apartado de \hyperref[sec-5]{Maven}.


\subsection{Introducción a Java}
\label{sec-2-1}
\label{fd70e36c-20a8-4577-90be-f671166ae3f1}


La sintaxis del lenguaje Java es muy parecida a la de C++. Se recomienda leer el documento \href{http://www.xtec.es/~acastan/textos/Java.pdf}{Introducción al Java para programadores C++}.


\subsection{Librerías Java}
\label{sec-2-2}

Java, a diferencia de C++, presenta casi desde sus comienzos una completa librería de clases que implementan muchas funcionalidades. De momento nos centraremosen las siguientes:

\subsubsection{Clases contenedoras}
\label{sec-2-2-1}

El manejo de clases contenedoras ofrecido por C++ mediante la STL presenta dos problemas principales: El confuso uso de los templates, y la poca utilidad de los mensajes de error. La librería Java contiene en el paquete \textbf{java.util.\*} un conjunto de clases contenedores estándar más fáciles de utilizar. Por ejemplo, Java ofrece las siguientes clases muy útiles:

\begin{itemize}
  \item \textbf{List} (listas), implementadas como  \emph{ArrayList} (\emph{arrays} en C++) o como \emph{LinkedList} (Listas enlazadas);
  \item \textbf{Map}, implementado como \emph{HashMap} (\emph{tabla hash});
  \item \textbf{Set} (colecciones sin repetición).
\end{itemize}

Aunque se suelen usar bastante las listas (\emph{List}), los \emph{Map}s son en ciertos casos más apropiados y permiten simplificar bastante el código. 

\subsubsection{Fechas}
\label{sec-2-2-2}

Es ampliamente reconocido el mal manejo de las fechas por parte de la librería estándar de Java, por tener un API poco intuitiva y nada cómoda, en el afán de realizar una biblioteca de fechas reutilizable para todos los calendarios, y no sólo el gregoriano.

El hecho de que una funcionalidad esté en un API estándar no impide que pueda utilizarse otra librería. En ese caso, se recomienda el uso de la librería \href{http://joda-time.sourceforge.net/}{Joda-Time}.

En el apartado \hyperref[sec-5]{añadiendo bibliotecas} se introduce cómo instalar las bibliotecas.

\subsubsection{Validaciones}
\label{sec-2-2-3}

En cualquier aplicación, es importante siempre validar los parámetros introducidos por el usuario y, especialmente, cuando haya que realizar conversiones, como por ejemplo, una fecha en formato de cadena.

Para ello es muy útil instalar la librería \textbf{commons} de la fundación Apache. En dicha librería, existe la clase \href{http://commons.apache.org/lang/api-2.5/org/apache/commons/lang/Validate.html}{Validate} que permite métodos que facilitan la comprobación de parámetros
(devuelve una excepción si no se cumple). La idea es similar al uso de \emph{assert} en C/C++.


%\subsection{Implementación de la funcionalidad}
%\label{sec-2-3}
%
%Para poder ofrecer libertad en el diseño, cada grupo debe de elegir la estructura de clases que crea más conveniente.
%Para poder probar el código de forma homogénea, se ofrecerá unos interfaces. Dentro del código deberá de existir
%clases que implementen dichos interfaces y que delegen las funcionalidades pedidas a las clases que implementen
%dichas funcionalidades. Se ofrecerá asímismo
%pruebas automáticas que comprueben la funcionalidad utilizando dicho interfaz.


\section{IDE}
\label{sec-3}
\label{ide}

Una gran ventaja es que existen para Java entornos de desarrollo que permiten trabajar de forma mucho más cómoda. Entre los distintos entornos destacan dos que son gratuitos y libres, ambos multiplataforma, con versiones para Linux, Windows y Mac (gracias a la portabilidad de Java). Ambos están fácilmente disponibles en la web. 

\begin{itemize}
\item \href{http://eclipse.org/}{Eclipse} es el IDE libre más popular. Existes muchas versiones de eclipse personalizadas para distintos propósitos, como por ejemplo \href{http://www.springsource.org}{SpringSource Tool Suite}, preparada para el desarrollo con el contenedor Spring y el framework Grails, basado en Java
\item \href{http://netbeans.org/}{Netbeans}, es el IDE oficial de Sun/Oracle,
\end{itemize}

%A su vez, en el campus virtual existe una transparencia sobre el entorno de desarrollo Eclipse, \emph{Introduccion\_{}eclipse.pdf}, introduciéndolo. 

\subsection{Netbeans y Eclipse}
\label{sec-3-1}

Ambos entornos son muy completos, y la elección de uno u otro depende en gran parte de preferencias personales. Eclipse posee una estructura  más modular (existen módulos para casi todo), y Netbeans es más compacto. Esto presenta ventajas y 
desventajas. 

Gracias a la herramienta \emph{maven} la estructura de los proyectos creados, independientemente del entorno utilizado, será la misma. Por tanto, la
compatilidad está garantizada, aunque es recomendable usar un mismo IDE para todos los miembros de un equipo de proyecto.

Asímismo, la existencia de Mylyn para la gestión de tareas hace Eclipse más recomendable para usar, pues ayuda no sólo en la gestión de la construcción del código, sino también en tareas de gestión del proyecto asociadas al proceso de desarrollo.
 
\subsection{Características comunes}
\label{sec-3-2}

Las funcionalidades más usuales de estos entornos son: Editar código fuente con resaltado de sintaxis, mantener abiertos múltiples ficheros de proyecto, encargarse de construir (compilar y ejecutar los programas) sin necesidad de crear un \emph{Makefile} o algo similar \emph{a mano}), pasar las pruebas automáticas que se hayan definido,\ldots{}.

Otra opción que ambos IDEs admiten es la inclusión de un depurador de código. Para poder terminar las prácticas correctamente y a tiempo \emph{es muy importante saber utilizar el depurador}. Es una herramienta básica de todo desarrollador, sin la cual se tardaría demasiado.

Ambos entornos ofrecen ciertas características avanzadas.

\subsubsection{Autocompletado}
\label{sec-3-2-1}

A menudo es pesado tener que recordar el nombre de las variables y/o métodos, y esa repetición es proclive a errores.  Esto es especialmente molesto en Java, ya que es un lenguaje bastante explícito.  Una funcionalidad que se vuelve necesario es un autocompletado inteligente. En ambos IDEs, al empezar a escribir una variable y pulsar \textbf{CTRL+SPACE}, se muestran los posibles nombres, a elegir. Si es únicamente uno, se completa. 

Con esta característica no hay excusas para no usar variables con nombre muy descriptivo, aunque sean más largas de escribir :-). 

Otra opción muy interesante del autocompletado es que cuando se desea llamar a un método de un objeto, basta con pulsar su nombre seguido del punto (\textsf{objeto.}) y pulsar \textbf{CTRL+SPACE}. Esto muestra los distintos métodos posibles
(métodos públicos de esa clase), incluyendo los parámetros necesarios. Es muy útil a la hora de saber las posibilidades que brindan las clases.

\subsubsection{Detección y resaltado dinámico de errores}
\label{sec-3-2-2}


Es una característica muy útil que evita muchas compilaciones. Conforme se va escribiendo, va detectando
posibles errores sintácticos, y los marca de color rojo. Así, una vez terminado un trozo de código (sentencia, 
función), la existencia de marca rojas indica errores a arreglar. De esta manera, se pueden arreglar
errores sin esperar a compilar el proyecto.
\subsubsection{Refactorización}
\label{sec-3-2-3}




A menudo es necesario hacer cambios en el código, no para aumentar la escalabilidad, sino para mejorar el diseño.
Ha estos cambios se les denomina refactorización. Es una técnica consistente en una serie de cambios pequeños, para
mejorar el mantenimiento y legibilidad del código. 


Dado que introducir un pequeño cambio (como cambiar el nombre de un método o clase) puede ser muy laborioso, los
IDEs actuales ofrecen utilidades para ello (por ejemplo, renombrar una clase o método cambiando automáticamente
todas sus referencias del proyecto. 
\subsubsection{Pruebas automáticas}
\label{sec-3-2-4}




Como vemos en el apartado de pruebas automáticas, para comprobar el correcto funcionamiento de la funcionalidad 
implementada, es necesario que el proyecto pase determinados casos de tests automáticos. 


Los IDEs permiten facilitar la creación de los casos de tests, creando el esqueleto, pudiendo centrarse en 
los métodos de tests. Tambien permiten facilitar su ejecución con una simple tecla, y comprobar el número de tests
pasados (y en qué ejemplos no se han pasado los tests).
\subsubsection{Gestión de librerías}
\label{sec-3-2-5}




Tal y como se ha indicado, se usará \hyperref[sec-5]{Maven} para instalar las librerías. Ambos IDEs permiten un uso muy sencillo. 


Antes de poder trabajar y/o crear el proyecto bajo Eclipse, es necesario instalar el plugin \href{http://m2eclipse.sonatype.org/}{M2Eclipse}, dicho
plugin, al igual que múltiples plugins, pueden instalarse directamente desde el propio entorno de Eclipse, 
sin problemas.


En primer lugar, es necesario crear el nuevo proyecto como un proyecto Maven, para poder usarlo. Si se crea el
proyecto como otro tipo de proyecto no podrá usarse, habrá que crear el proyecto de nuevo. 


Un vez creado el proyecto, se puede incluir dependencias de las librerías que queremos, indicando incluso el
número de versión mínimo requeridad. Ambos IDEs ofrecen buscar las librerías. 


Por defecto, ambos poseen una serie de repositorios, pero puede aumentarse esa lista para poder instalar
librerías adicionales (todas las librerías recomendadas en esta guía poseen un repositorio maven). El concepto
es muy similar a añadir un nuevo repositorio Ubuntu usando la aplicación \emph{Origenes del Software}, o directamente
usando \emph{apt-add-repository}. Para acceder al repositorio sólo hay que seguir los enlaces de las librerías
de la documentación.


En los enlaces se puede ver cómo se puede \href{http://m2eclipse.sonatype.org/adding-project-dependencies-in-m2eclipse.html}{añadir librerías en Eclipse} y \href{http://wiki.netbeans.org/MavenBestPractices}{añadir librerías en Netbeans}
usando Maven. 
De todas maneras, se explicará la instalación de las librerías en prácticas.
\subsubsection{Herramientas de trabajo en grupo}
\label{sec-3-2-6}




Los entornos permiten facilidades de trabajo en grupo, como el soporte de los sistemas de control de versiones
\hyperref[svc]{comentados} en la documentación. En el siguiente \hyperref[soporte_svn]{apartado} se detalla cómo se usa.
\section{JUnit: Pruebas automáticas}
\label{sec-4}




Como veremos en clase, es necesario para tener un cierto nivel de confianza en el código realizar un conjunto
de pruebas. Aunque inicialmente se suele hacer \emph{a mano} (es decir, invirtiendo horas haciendo pruebas con
valores introducidos a mano y comprobados los resultados). Esa forma de hacer pruebas es intuitiva, pero es
demasiado laboriosa, y no se suelen repetir. Por tanto, cuando al modificar el código e introducir errores, éstos
no suelen ser detectados. 


Otra forma de abordar las pruebas es mediante el empleo de programas automáticos que realicen tests. Esto permite
probar de forma periódica la funcionalidad implementada, y tener ejemplos directos cuando no funciona de forma
correcta (es más fácil de depurar al tener ona región pequeña de código que no funciona correctamente). 


En Java, esta funcionalidad se implementa por medio de la librería \href{http://junit.sourceforge.net/}{JUnit}, que está muy documentada en la
Red, y con \href{http://www.slideshare.net/tom.zimmermann/unit-testing-with-junit}{distintos ejemplos}. De todas formas, invertiremos una sesión práctica con las pruebas automáticas.


La idea es el de crear clases adicionales con distintos métodos encargados de probar el código. \textbf{JUnit} se diseñó
para crear pruebas de unidad, por lo que es común tener por cada clase que ofrece una clara funcionalidad, una
clase de Tests, asegurando que el resultado sea el esperado, para distintas situaciones.  
\section{Maven: Añadiendo Librerías}
\label{sec-5}
\label{maven}




Un primer problema a la hora de trabajar con un IDE y/o con un \emph{framework} (como los \emph{frameworks webs} para Java) es que cada uno de ellos implica una determinada estructura de ficheros y
directorios, haciendo difícil compilar y/o desarrollar sin tener dicho entorno.


Otro problema que se presenta es el de la instalación de librerías, problema muy dependiente de las librerías. 


Ambos motivos obligaban a tener que usar el mismo IDE con el que se desarrolló para compilar el
proyecto. Para resolverlo surgió un estándar de compilación, denominado \textbf{ant} (equivalente al uso de
\textbf{make} para C/C++). Dicho programa, parte de una definición en un fichero \emph{xml} denominado
\textbf{build.xml} que permite establecer las dependencias.


Con el uso de \textbf{ant}, mejoró el proceso, pero no se resolvió el problema de una estructura común ni
facilitar la instalación de librerías.


\href{http://maven.apache.org/}{Maven} surgió como una solución a este problema. Maven es un sistema que ofrece las mismas funcionalidades
que \textbf{ant} (lo llama internamente) pero presenta grandes ventajas.


Explicar el uso de \textbf{Maven} excede las pretensiones de esta guía, por lo que paso simplemente a comentar dos
aspectos importantes para el desarrollo de la práctica.
\subsection{Definición de la estructura}
\label{sec-5-1}




Al crear el proyecto como un proyecto Maven, el propio sistema se encarga de crear una estructura de ficheros
y directorios que es estándar. Por tanto, una vez creado, es posible compilarlo sin necesidad de tener el IDE instalado, 
únicamente con Maven. 


A su vez, aunque no lo veremos en la práctica, el Maven también permite definir fácilmente estructuras necesarias para
desarrollar en determinados \emph{frameworks}, facilitando mucho su desarrollo (puede generar el esqueleto de la casi
totalidad de \emph{frameworks web}).


\href{http://www.youtube.com/watch?v%3DX8lu7Oi23YQ}{Ejemplo en video usando m2eclipse}.
\subsection{Instalar librerías}
\label{sec-5-2}




A cualquier usuario de Linux agradece la facilidad de instalar aplicaciones directamente 
por medio de un conjunto de repositorios. Esto permite no sólo un lugar en donde encontrar la aplicación, si no que
también permite, a la hora de desarrollar una aplicación que dependa de
una librería, poder indicar dicha dependencia. De esta forma, cuando se solicita la instalación de la aplicación, 
automáticamente se descarga e instala también las librerías de las que depende, haciendo el proceso de 
instalación muy sencillo tanto para el usuario como para el/la desarrollador/a. 


Maven permite realizar esta misma tarea. Por medio del sistema Maven (se puede hacer desde el propio IDE) se puede
indicar que una aplicación requiere una determinada librería. Así, a la hora de compilar y/o ejecutar el programa, 
maven instalará las librerías si no están ya instaladas, por medio de repositorios externo. 


Esto permite que a la hora de distribuir un proyecto \emph{maven} sólo sea necesario distribuir vuestro código, sin
preocuparse de las librerías.


Además, permite indicar la versión de las librerías, evitando cualquier tipo de problema de versiones.


Esto, aunque algo más complejo desde un punto de vista técnico, con el uso de IDEs que soportan este sistema 
(se puede ver su uso, muy similar, tanto en \href{http://m2eclipse.sonatype.org/adding-project-dependencies-in-m2eclipse.html}{Eclipse} como en \href{http://wiki.netbeans.org/MavenBestPractices}{Netbeans}), acaba siendo más sencillo que instalar las 
librerías a mano. 


En prácticas vamos a ver cómo crear un proyecto maven e instalar dependencias. Para un primer vistazo, podemos
ver en el siguiente \href{http://www.youtube.com/watch?v%3DH8QdjyCB8Nw}{video de adición de dependencias en ma2eclipse.}
\section{Persistencia}
\label{sec-6}
\label{librerias}




No serviría de nada una aplicación que no almacenase las tareas/citas para que estuviesen disponibles en las 
siguientes ejecuciones. 


Un enfoque directo sería el uso de una Base de Datos (BD), y guardar los datos en ella.
Afortunadamente, en el API de Java se ofrecía un interfaz común para las distintas bases de datos, por lo que
toda librería que funcione bajo Java es independiente de la Base de Datos que tengamos instalada, pudiendo 
cambiar una por otra simplemente cambiando la configuración. Es decir, no existen dependencias innecesarias
entre la Base de Datos concreta y el código de la aplicación. 


Existen varios tipos de bases de datos, para distinto tipo de aplicaciones. 




\begin{itemize}
\item Por un lado, las aplicaciones simples pueden usar bases de datos empotradas (en la que es la
  propia librería de Java la encargada de gestionar los datos, sin necesidad de instalar un proceso
  de Base de Datos).
\item Por el otro, las aplicaciones más complejas, o web, requieren un sistema de BD relacional externo, que
  permite el compartir información entre aplicaciones.
\end{itemize}


En el caso de BD relacionales, se puede acceder directamente a la Base de Datos haciendo uso del
SQL, o por medio de una herramienta ORM, que permita asociar una clase con una tabla de la Base de
Datos, y permite recuperar y almacenar los objetos en la base de datos. Un ejemplo de un ORM muy
popular bajo Java es el \href{http://hibernate.org}{Hibernate}.
\subsection{Conceptos generales}
\label{sec-6-1}




Aunque existen muchos tutoriales libremente disponibles, no está de más introducir algunos conceptos
asociados para el hibernate.


\begin{center}
\includegraphics[width=.6\textwidth]{hibernate.png}
\end{center}




En primer lugar, la idea de hibernate es permitir almacenar y recuperar una clase Java en una base de
datos relacional. Permite guardarlas en distintos tipos de Bases de Datos, aunque para las pruebas
usaremos el MySQL para trabajar (aunque para depurar podemos una Base de Datos empotrada, ya que
es más rápida).


\begin{center}
\includegraphics[width=.6\textwidth]{esquemahibernate.pdf}
\end{center}




\begin{itemize}
\item Como se ve, la aplicación/librería utiliza un interfaz estándar JPA (Java Persistence API) que es
  interpretada por Hibernate, que es la que se conectará finalmente con la Base de Datos.
\item La aplicación será configurada por otra librería denominada Spring, ya que ofrece un método de
  configuración más flexible y cómodo.
\end{itemize}
\subsection{Definición de clases ORM}
\label{sec-6-2}




Para poder guardar una clase en una Base de Datos, es necesario establecer la correspondencia entre
los atributos de dicha clase y las columnas de las tablas en donde se almacenan. Existen dos formas
de proceder. Por un lado, el formato clásico era definir el objeto sin ningún tipo de información y
en un fichero xml aparte (persistence.xml) almacenar dicha relación. Sin embargo, la tendencia actual
(y nuevo estándar) es añadir una serie de términos (empezados por @) en el propio fichero Java que
permite indicar dicha correspondencia. 


La clase no tiene por qué heredar de ninguna clase particular, y debe poseer los atributos a guardar
con sus métodos de acceso correspondientes. Para que el sistema almacene la clase hay que añadir los
siguientes atributos:




\begin{description}
\item[@Entity] Define que la clase correspondiente se almacenará en la Base de Datos.
\item[@Table(name=''\emph{tablename}'')] Define el nombre de la tabla asociada a dicha entidad (si no se indica
  será el nombre de la clase).
\end{description}


Luego, antes de la declaración de cada atributo se pueden indicar propiedades.




\begin{description}
\item[@Id @GeneratedValue] Permiten definir el atributo siguiente como clave primaria de la tabla, se
     autogenerará su valor con cada nuevo objeto.
\item[@Column(name=''\emph{name}'', nullable=true/false)] Permite indicar el nombre de la columna que
     guarda el valor, permitiendo indicar si se admite valor nulo o no en la Base de Datos.
\item[@Basic(optional=true/false)] Permite indicar al ORM si el atributo puede ser nulo o no.
\end{description}


Un par de ejemplos:


\begin{lstlisting}
@Entity
@Table(name="author")
public class Author {
        @Id @GeneratedValue long id;
        @Column(name="name", nullable=false) @Basic(optional=false)
        private String name;
        @Column(name="country", nullable=false) @Basic(optional=false)
        private String country;
        ...
}
\end{lstlisting}




¬øCómo se puede reflejar la relación entre tablas? Evidentemente, por medio de referencias cruzadas. Para
indicar que un atributo debe de obtenerse a partir de otra tabla es necesario añadir otro tipo de información. 
Los atributos son los siguientes:




\begin{description}
\item[OneToMany(cascade=ALL)] Permite definir que el atributo siguiente (contenedora) es una
     referencia. El parámetro \emph{cascade} permite hacer borrados/modificaciones en cascada.
\item[ManyToOne(cascade=ALL, mappedBy=''otherclase'')] Permite indicar que el atributo indicado
     (anónimo) referencia a otra clase. mappedBy es opcional permite indicar el atributo de la otra
     clase (si la relación es biyectiva).
\end{description}


Se puede consultar más información en el \href{https://en.wikibooks.org/wiki/Java_Persistence/Relationships}{Wikibooks sobre relaciones usando JPA}. 
Un ejemplo sencillo sería indicar que un libro puede tener un (único) autor se reflejaría de la siguiente forma.  


\begin{lstlisting}
@Entity
@Table(name="book")
public class Book {  
        ...
        @ManyToOne(optional=false, cascade=CascadeType.ALL)
        private Author author;
        ...
}
\end{lstlisting}
\subsection{Uso de la Base de Datos}
\label{sec-6-3}




Para acceder a la Base de Datos, una vez definidas las clases que vamos a relacionar en la Base de Datos, necesitamos
un objeto \textbf{session} de tipo SessionFactory (en la \hyperref[sec-6-4]{configuración} vemos cómo obtenerlo). 
\subsubsection{Recuperar un objeto}
\label{sec-6-3-1}




Si tenemos el \emph{id} de un objeto, podemos recuperarlo simplemente con \emph{session.load(Class, id)}. ejemplo:
session.load(Author.class, id).


En el caso más frecuente no tenemos el \emph{id}, pero sí un conjunto de restricciones, y podemos obtenerlas mediante
la instrucción \textbf{createQuery} que posee la siguiente sintaxis:


\begin{lstlisting}
session.createQuery(from XXX as XX where XXX.yy ...)
\end{lstlisting}


Esto permite definir una consulta. A la hora de poner la condición no se debe de crear una cadena con los valores
concretos, ya que puede generarse todo tipos de problemas. La opción es uso de parámetros (empezados por el símbolo `:'). 


Es decir, en vez de hacer
\begin{lstlisting}
Query q = sess.createQuery("from DomesticCat cat where cat.name = " +name);
\end{lstlisting}


Se debe de hacer de la siguiente forma:


\begin{lstlisting}
Query q = sess.createQuery("from DomesticCat cat where cat.name = :name");
q.setString("name", name);
\end{lstlisting}


Los métodos para asignarle valores a los parámetros son: \emph{setString}, \emph{setDate}, \ldots{} Se puede consultar la documentación.


Una vez asignado valores a los parámetros, es necesario indicar que se devuelva los resultados. Se puede indicar que se
devuelva un único resultado, con uniqueResult (si existe sólo un elemento, devuelve null si no lo encuentra).


\begin{lstlisting}
...
q.setString("name", name);
Cat cat = q.uniqueResult();
\end{lstlisting}


Otra opción es indicar que se desea un grupo de objetos, mediante el método \textbf{list}. 


\begin{lstlisting}
...
q.setString("race", race);
List<Cat> cats = q.list();
\end{lstlisting}
\subsubsection{Modificar y guardar un objeto}
\label{sec-6-3-2}




Se modifica recuperando el objeto de la Base de Datos, modificando los atributos que queramos, y volviéndolos a almacenar
en la Base de datos. 


Un ejemplo tonto, que cambia para todos los gatos de raza `angola' por `Angola':


\begin{lstlisting}
List<Cat> cats = session.createQuery("from DomesticCat cat where race = :race").setString("race", "angola").list();


for (Cat cat : cats) {
    cat.setRace("Angola");
}


session.save(cat);
\end{lstlisting}
  
Para borrar, existe el método delete(). 
\subsection{Configuración}
\label{sec-6-4}
\label{cfgorm}




Dado que se ofrece un ejemplo en el campus virtual, nos centramos en explicar los componentes que
deben o pueden personalizarse.  


Como en todo proyecto Maven la descarga instalación de las librerías Hibernate y Spring se realiza de forma
automática por medio del fichero \emph{pom.xml}. 


La configuración de la configuración de la Base de Datos se hace por medio del fichero \textbf{beans.xml}
en el directorio \emph{resources}.


\lstset{emph={property,bean,value,class,id,name}}
\begin{lstlisting}
<bean id="myDataSource" class="org.apache.commons.dbcp.BasicDataSource" 
    destroy-method="close">
    <property name="driverClassName" value="com.mysql.jdbc.Driver"/>
    <property name="url" value="jdbc:mysql://localhost/databasename"/>
    <property name="username" value="usuario"/>
    <property name="password" value="clave"/>
  </bean>
\end{lstlisting}


Para adaptarlo a otra configuración sólo hay que cambiar las siguientes propiedades: 




\begin{description}
\item[driverClassName] Contiene el nombre de la clase que conecta con la Base de Datos. No hay que
     cambiarla si se usa el mysql, sí su se desea usar otra Base de Datos (consultar la
     documentación de dicha BD).
\item[url] La url contiene la información de conexión, puede depender de la base de datos. En este
         caso sólo se cambiaría el nombre de la base de datos (\emph{database}). Evidentemente,
         \emph{localhost} es el servidor que contiene la Base de Datos.
\item[username] Usuario con el que se va a acceder a la base de datos (un usuario \emph{normal}, no \emph{root},
              pero con todos los permisos para la base de datos indicada).
\item[password] Contraseña del usuario de la base de datos.
\end{description}


Adicionalmente, hay que indicar las clases que se relacionarán con la Base de Datos (y que deberán 
de poseer los decoradores @Entity, \ldots{}). Esto se realiza añadiendo el nombre completo de las clases en la lista
de la propiedad \textbf{annotatedClasses} del fichero \emph{beans.xml}. Un posible ejemplo sería el siguiente:


\lstset{emph={property,name,list,value}}
\begin{lstlisting}
<property name="annotatedClasses">
        <list>
                <value>org.uca.dss.example.data.Author</value>
                <value>org.uca.dss.example.data.Book</value>
                ...
        </list>
</property>     
\end{lstlisting}


Por último, necesitamos que las clases que nos permiten operar con la Base de Datos contengan un
objeto de tipo \textbf{SessionFactory} para poder comunicarse con la Base de Datos (usando el interfaz JPA). 
¬øY cómo se puede iniciar ese objeto? Spring es capaz de crear objetos con dicho atributo inicializado. 
Para ello, es necesario indicarlo en el fichero \emph{beans.xml} con una notación como la siguiente:


\lstset{emph={bean,id,property,name,ref,class}}
\begin{lstlisting}
<bean id="autores" class="org.uca.dss.example.dao.RealAuthors">
    <property name="sessionFactory" ref="mySessionFactory"/>
  </bean>
  <bean id="libros" class="org.uca.dss.example.dao.RealBooks">
    <property name="sessionFactory" ref="mySessionFactory"/>
  </bean>
\end{lstlisting}


De esta manera, el atributo \textbf{sessionFactory} estará adecuadamente iniciado para los objetos
creados autores y libros de las clases correspondientes. Para acceder a dichos objetos dentro
del código sólo es necesario hacer:


\lstset{language=java}
\begin{lstlisting}
ClassPathXmlApplicationContext ctx = new ClassPathXmlApplicationContext("beans.xml");           
Authors autores = (Authors) ctx.getBean("autores");
Books libros = (Books) ctx.getBean("libros");
\end{lstlisting}


en donde Authors y Books son interfaces (que las clases RealAuthors y RealBooks cumplen).
\section{Trabajando en equipo, usando una forja}
\label{sec-7}




Para poder trabajar en equipo es necesario el uso de una serie de herramientas que permita trabajar sobre el 
mismo proyecto, y evitar tener posibles conflictos. 


Lo primero que hay que hacer es asegurarse que se trabaja sobre el mismo código, para lo cual es necesario
un \hyperref[sec-7-2]{Sistema de Control de versiones}. Otras opciones que nos permiten es mantener un \textbf{wiki} en el que documentar
y un \hyperref[sec-7-4]{sistema de tickets}. Este tipo de herramientas son especialmente importantes ya que permiten trabajar de forma
físicamente separada.
\subsection{Configurando un proyecto}
\label{sec-7-1}




Para permitir trabajar físicamente separado es necesario que toda la información del proyecto esté disponible en un
servidor conectado a internet para permitir acceder a sus miembros. Afortunadamente, existen distintos servidores, forjas, 
libremente disponibles. Un ejemplo sería la forja universitaria \href{http://forja.rediris.es}{de red iris}. También existen distintas forjas para otros
sistemas de control de versiones, como \href{http://github.com}{Github} o \href{http://bitbucket.org}{BitBucket}. 


En nuestro caso, dado que vamos a usar el Subversion, y queremos usar un sistema de tickets, vamos a usar la página web
\href{http://www.assembla.com}{assembla} que nos permite usar ambos, con un interfaz relativamente sencilla. 


El proceso es el siguiente:




\begin{enumerate}
\item Acceder a la web de assembla: \href{http://www.assembla.com}{http://www.assembla.com}
\item Elegir crear un espacio: Cree un Espacio -> Cree un Proyecto
   Público -> Hosting de Subversion con Tickets Integrados.
\item Inscribirse como usuario de Assembla.
\begin{enumerate}
\item Darse de alta como usuario.
\item Elegir como id dss2012-grupoX-nombre, identificándose como
      grupo, y un nombre corto del proyecto.
\end{enumerate}
\item Acceder a la url del proyecto.
\end{enumerate}
\subsection{Compartiendo código: Sistema de Control de versiones}
\label{sec-7-2}
\label{scv}




Como podrán notar, no es lo mismo desarrollar código en separado (como están más acostumbrados) a hacerlo en equipo. 
Además de las múltiples dificultades comunicativas y organizativas, se añade la dificultad de trabajar sobre el mismo
código. 


Si no se trabaja sobre el mismo código, es difícil trabajar, ya que hay que estar enviando los ficheros cambiados, con
el riesgo de olvidar enviar clases cambiadas, o hacer cambios incompatibles entre sí, con lo que se puede generar
situaciones muy difícil. 


Sin embargo, desde hace muchos años existe una solución, utilizar un servidor que guarde vuestro código, y sincronizar
vuestro código con dicho servidor. Existen sistemas que permiten esto, denominados \textbf{Sistemas de Control de Versiones}
(estoy simplificando mucho como detectará aquel ó aquella que tenga conocimiento previo de estos sistemas). 


De esta manera, se puede subir siempre todos los cambios realizados, y tener la confianza de trabajar con el mismo 
código. 


El uso de un Sistema de Control de Versiones permite: 




\begin{itemize}
\item Mantener centralizado el código, evitando el riesgo de que cada desarrollador tenga una versión diferente.
\item Que los desarrolladores puedan acceder siempre al código actualizado.
\item Mantener todas las versiones, y no sólo la última versión. De esta manera se permite \emph{volver atrás} si hay un
  error.
\item Los desarrolladores pueden bajarse el código actual, aplicar cambios, y subir sus cambios al repositorio.
\end{itemize}


La idea de trabajo es la siguiente:




\begin{itemize}
\item Primero, se descarga el código del servidor.
\item Se hacen cambios locales, hasta tener una versión más avanzada, que al menos compile.
\item Se sube el código cambiado al servidor.
\end{itemize}


Además, en todo momento, se puede descargar la nueva versión del servidor, detectando posibles cambios incompatibles
(colisiones), en cuyo caso avisa de los cambios incompatibles. 


Como se trabaja con los cambios, varios desarrolladores pueden, por ejemplo, añadir distintos métodos a la misma clase, 
sin que se produzca ningún problema, lo cual simplifica mucho el trabajo de integración. Un cambio incompatible sería
que dos desarrolladores cambiasen simultáneamente la misma sentencia, o bloque. 


En caso de detectar cambios incompatibles, el seguno desarrollador recibiría un aviso al intentar \emph{subir} su código, 
y deberá de adaptar su cambio al cambio anterior.


Hay que recordar que un sistema de versiones espera un cierto grado de organización entre los desarrolladores, no es una
herramienta de sincronización, simplemente garantiza coherencia en el código.


Otra opción que permite un sistema de control de versiones, es que guarda todas las versiones, no sólo la última
versión, con lo que siempre se puede volver a una versión anterior en el caso de cometer un error en el desarrollo,
quitando el \emph{miedo al cambio}. Además, esto implica que no hay que desactivar código poniéndolo entre comentarios, si un
código no se utiliza se debe de borrar, ya que siempre se podria recuperar si es necesario (aunque no suele ser
necesario).
\subsubsection{Acceder al repositorio del proyecto}
\label{sec-7-2-1}




Dentro de la web del proyecto se indica una URL que permite indicar al subversion dónde se encuentra el repositorio remoto. 
Un ejemplo sería la URL \href{http://subversion.assembla.com/svn/dss2012-hibernate-ejemplo/}{http://subversion.assembla.com/svn/dss2012-hibernate-ejemplo/}


Dentro de esa URL existe un fichero README y un directorio \emph{trunk}. El código fuente de nuestra aplicación debe estar en dicho
directorio. En nuestro caso, no se guardará únicamente el código fuente, si no también los ficheros de configuración
(\emph{pom.xml}, \emph{beans.xml}, \ldots{}). El propio entorno IDE es capaz de trabajar usando el SCV, por lo que se encargará de añadir
dichos ficheros al repositorio. 
\subsubsection{Soporte del IDE de los SCV}
\label{sec-7-2-2}




Existen múltiples sistemas de control de versiones, pero los más utilizados son \href{http://www.subversion.org}{Subversion} (modelo centralizado) 
y el \textbf{Git} (modelo de desarrollo distribuido). En este trabajo vamos a hacer uso del \textbf{Subversion}. 


Ambos están soportados por los IDEs recomendados, pero hay que recordar instalar el plugin Subeclipse para poder usarlo
(se instala directamente desde la configuración del eclipse, al igual que con el otro plugin).


Para entender su uso, mejor que mirar directamente un tutorial de Subversion (que explicará
cómo usarlo bajo linea de comandos), es mejor aprender cómo se puede usar en los IDEs (obteniendo una visión más directa).
En los enlaces siguientes hay un \href{http://www.ibm.com/developerworks/opensource/library/os-ecl-subversion/}{tutorial de Subversion bajo Eclipse} y otro \href{http://netbeans.org/kb/docs/ide/subversion.html}{tutorial de Subversion bajo Netbeans}.
\subsection{Documentando con Wiki}
\label{sec-7-3}




Dado que la documentación de una aplicación (incluyendo la documentación de diseño) es algo vivo, ¿qué mejor forma
que usar la wiki del proyecto para documentarlo?


En el wiki se va a copiar la documentación entregada en la primera entrega, y se modificará los cambios sobre él
(incluso los diagramas), permitiendo de ese modo tener un histórico de la documentación.
\subsection{Sistema de Tickets}
\label{sec-7-4}
\label{tickets}


Por hacer.


\end{document}